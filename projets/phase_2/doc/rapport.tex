\documentclass[a4paper,12pt]{article}
\usepackage[utf8]{inputenc}
\usepackage[french]{babel}
\usepackage{amsmath}
\usepackage{amssymb}
\usepackage{geometry}
\pagenumbering{gobble}
\geometry{margin=2.5cm}

\title{Rapport Projet Phase 2 (LDP)}
\author{PROPS Thibaut}
\date{9 Avril 2025}

\begin{document}

\maketitle

\section*{Les fonctions \texttt{parse}}

Soit $n$, la position du mot que l'on veut analyser.  
On calcule la position des $n + 1$ espaces, puis on prend la sous-chaîne de la chaîne d'instruction située entre le $n^\text{ième}$ et le $(n+1)^\text{ième}$ espaces pour extraire le mot souhaité.

\section*{Fonction \texttt{is\_register()}}

On vérifie si le prétendu nom de registre est bien une chaîne de caractères de longueur 1 et que c’est une lettre entre \texttt{a} et \texttt{d} (c’est-à-dire que sa représentation numérique se situe entre celle de \texttt{a} et celle de \texttt{d}).  
Cette fonction permet de savoir, lorsque l'on veut connaître la valeur d’un objet, s’il s’agit d’une \texttt{l-value} ou d’une \texttt{r-value}.

\section*{Fonction \texttt{write()}}

Étant donné que les cases ne peuvent contenir que des nombres de 8 bits, on va découper le nombre en entrée en deux via la formule :

\vspace{0.5em} 

\begin{itemize}
  \item \texttt{lower8 = value \& 0xff} : ne prendre que les 8 premiers bits du nombre.
  \item \texttt{upper8 = value >> 8} : décale le nombre vers la droite. Et vu que nous sommes sur des entiers, cela supprime les 8 premiers bits et transforme les 8 derniers pour obtenir un nombre que l’on peut représenter sur 8 bits.
\end{itemize}

\vspace{0.5em} 

Ainsi, on écrit dans la première case \texttt{lower8} puis dans la deuxième \texttt{upper8} (soit une représentation en \textit{Little-Endian}).

\section*{Fonction \texttt{read()}}

On va lire deux nombres 8 bits consécutifs pour reconstruire le nombre 16 bits via la formulation \textit{Little-Endian}.

\section*{Fonctions \texttt{push()} et \texttt{pop()}}

On va simplement appeler les fonctions \texttt{write()} et \texttt{read()} tout en mettant à jour la variable \texttt{stack\_pointer} afin d’avoir un \textit{ADT} de pile.

\end{document}