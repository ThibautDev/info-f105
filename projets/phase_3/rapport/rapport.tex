\documentclass{article}
\usepackage[top=1cm, bottom=2cm]{geometry}
\usepackage[strings]{underscore}

\title{Rapport projet INFO-F105 - LDP1: phase 3}
\author{PROPS Thibaut}
\date{}

\begin{document}

\maketitle

\section{Questions}

\subsection{À quoi sert le dernier opérateur de \texttt{class Register} ?}

L'opérateur \texttt{uint16_t()} sert retourner la valeur de l'élément \textbf{privé} \texttt{_value} lorsque à appeler la élément classe \texttt{Register} comme \texttt{uint16_t()}.

\begin{verbatim}
    Register a; // _value = 0 by default
    uint16_t val = a; // Call operator uint16_t()
\end{verbatim}


\subsection{Essayez de le retirer puis recompilez votre programme. Y a-t-il quelque chose qui a changé ?}

Oui, étant donné que \texttt{_value} est un élément \textbf{privé}, il n'y a aucun moyen d'accédé à celui ci. Cependant si \texttt{Register} avait été un \texttt{struct} au lieu d'une \texttt{class}, nous n'aurions pas eu de problème d'accessibilité il serait possible d'accédé à \texttt{_value}.

\begin{verbatim}
    Register a; // _value = 0 by default
    uint16_t val = a._value; // Call directly _value
\end{verbatim}

\subsection{Que cela changerait-il si \texttt{Instruction} était une \texttt{class} plutôt qu'une \texttt{struct} ?}

Si \texttt{Instruction} avait été une \texttt{class}, alors l'access aux éléments \texttt{opcode} et \texttt{operands} n'aurait pas pus se faire directement via \texttt{Instruction -> opcode}. On aurait du créé 2 \texttt{getter} et 2 \texttt{setter} pour pouvoir y acceder.

\section{Choix d'implémentation notables}

\end{document}